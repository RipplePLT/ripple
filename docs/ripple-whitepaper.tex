\documentclass{article}
\usepackage[utf8]{inputenc}
\usepackage{indentfirst}
\usepackage{listings}
\usepackage{fullpage}
\usepackage{tikz}
\usepackage{tikz-qtree}
\usepackage{tipa}
\usetikzlibrary{arrows,automata}
\usepackage{wrapfig}

\lstset{
numbers=left,
numberstyle=\small,
numbersep=8pt,
frame = single,
captionpos=b,
framexleftmargin=15pt}

\title{Ripple \\ {\large\it make waves}}
\author{\begin{tabular}{rcl}
\textbf{Project Manager:}    &   Amar Dhingra   &   \textbf{asd2157} \\
\textbf{System Architect:}   &   Alexander Roth &   \textbf{air2112} \\
\textbf{System Integrator:}  &   Artur Renault  &   \textbf{aur2103} \\
\textbf{Language Guru:}      &   Spencer Brown  &   \textbf{stb2132} \\
\textbf{Tester \& Verifier:} &   Thomas Segarra &   \textbf{tas2172}
\end{tabular}}
\date{2015 -- 02 -- 25}

\newcommand{\code}{\texttt}
\begin{document}

\maketitle

\section*{Motivation}
With the proliferation of devices that are constantly connected today,
information becomes outdated faster than ever. Keeping up with rapidly changing
data is therefore increasingly problematic in software development. Ripple aims
to fulfill this need by enabling programmers to work natively with streams of
external data, using a simple and intuitive syntax. Ripple will be a valuable
tool in any field, but we expect that it will have especially important
applications in statistics, scientific research, and finance.

\section*{Why Ripple?}
Ripple is designed to make working with dynamic data easier. Data comes into
a Ripple program through ``streams'' from external sources. Dealing with this data is straightforward in Ripple because the language is reactive: changes in a variable propagate, or ``ripple,'' through the program to affect other variables linked to it.
Variables that depend on external data are then as simple to deal with as
variables defined within the program itself.

Consider a program that displays the current weather based on periodic
updates from the internet. In most existing languages, the programmer would
repeatedly need to:
\begin{enumerate}
\itemsep0em
\item create a socket,
\item connect to a server,
\item make an HTTP request,
\item receive a response,
\item parse the payload,
\item update the display to include the new information.
\end{enumerate}
In Ripple, one only needs to open a stream to the web server, and then link the
stream to a variable; the variable will automatically reflect all relevant
changes.
    
Ripple's value becomes clearer when we consider data that must be processed
after it enters the program. In most cases, the data we receive is not exactly
in the form we need, and we have to perform additional operations to derive real
value from it. If, for example, the weather stream produced temperatures in
Fahrenheit, we would need an additional derivation if we wanted to display it in
Celsius. In Ripple, this is as simple as \emph{linking} a Celsius variable to
the weather stream, and applying the Fahrenheit-Celsius conversion formula. The
direct correspondence between the coded variable definition and the mathematical
formula makes the language eminently readable.

\section*{Hello, world!}
This simple program begins by printing "Hello world!" Then it assigns
\code{w} a filestream, which determines the file via command line arguments. It
reads a file line by line and every time the program reads a new line, the value
of \code{w} changes to the last line read, and the print statement within the
link statement executes. Essentially, the program says hello to the world, and
to every line in the file.
\begin{lstlisting}[title=\emph{hello.rpl}]
#include <FileStreamReader>;

func main(list<String> args) {

    String filename = args[1];

    String h = "Hello ";
    String w = "world!";

    print(h,w);

    link(w -> FileStreamReader(filename)) {
        print(h,w);
    }
}
\end{lstlisting}

\section*{What is Ripple?}

\subsection*{Ripple is reactive}
In a typical imperative language, the statement \code{x = y + z} immediately
sets x to the current sum of y and z; if \code{y} or \code{z} changes
afterwards, \code{x} is not updated unless the programmer explicitly specifies
when to do so. In a reactive language like Ripple, if the variables are
\emph{linked}, \code{x} updates every time either \code{y} or \code{z} changes.
Thus, the programmer is not required to manually and repeatedly update related
variables.

\subsection*{Ripple is connected}
From our previous example, the values of \code{y} and \code{z} do not need to live within tthe program: both can be dynamically retrieved from a file, a device, or the
Internet in the form of streams. Ripple abstracts away the socket and file I/O
involved in these operations, eliminating a substantial amount of boilerplate.
This encapsulation is achieved by providing a library of pre-defined \code{StreamReader}s for common input types, and an interface that allows developers to easily build their own \code{StreamReader}s. Developers will thus have the flexibility to handle a variety of input formats, turning data of any kind into a Ripple-compatible stream.

\subsection*{Ripple is simple}
It only takes a single line of Ripple code to \emph{link} a data stream to a
variable, which would often be a long and cumbersome process in other languages. This syntax makes it easy to understand how changes in a single variable affect the state of the entire program, and dramatically simplifies what might otherwise be a prohibitively complex control flow. In addition, Ripple is statically-typed,
ensuring compatibility between variables and the streams they are \emph{linked}
to, thereby minimizing unexpected behavior.

\section*{What else is out there?}
Existing implementations of the reactive programming paradigm are platform-
dependent. For example, Bacon.js and Elm rely on JavaScript, and were written to
simplify the construction of user interfaces on the Web. ReactiveCocoa, to name
another, is targeted only at iOS and OSX. Ripple, by contrast, is platform-
independent. Any machine that can compile ISO C++11 code can compile Ripple.
    
Similarly, most existing reactive languages are designed specifically to react to a
user's input; Ripple is built to react to anything for which a
\code{StreamReader} can be defined. This generalization significantly broadens the class of problems that can be solved by a reactive program.

\section*{Target Audience}
Ripple targets developers with knowledge of a C-like programming language who work with dynamic information. We can see
Ripple being used in fields ranging from data science, statistics, natural
language processing, machine learning or any other field where analyzing data
over time is important.

\section*{Syntax}
Ripple is meant to be simple to pick up to anyone who has experience
programming in any C-like language. Thus, Ripple retains the familiar control
flow statements from C, such as \code{if-else} statements, \code{for}-loops, and
\code{while}-loops. Similarly, it retains all the standard data primitives from
C, with the addition of \code{byte} and \code{string} types.
    
The \textbf{link} keyword creates relationships between variables and
either:
\begin{enumerate}
\item data streams, or
\item other variables through \emph{chaining}
\end{enumerate}

\emph{Chaining} is the process in which we link one variable to $n$ other
variables, such that there cannot exist a loop between any pair of variables. That is, the program creates a minimum-spanning tree of connections for the variables,
where the root variable is the first linked variable, and the leafs are the last
linked variables. 

To illustrate this point, suppose we have the program:

\begin{lstlisting}[title=\emph{sample.rpl}]
...

int x = 5;
int y, z, q, v;

link (x) {
    link (y -> x + 2) {
        link (z -> y + 10) {}
        link (v -> y + 42) {}
    }
    link (q -> x - 4) {}
}
...
\end{lstlisting}

In this program, we have initialized the integer variable \code{x} with a value of 5. Afterwards, we define four more integer variables: \code{y}, \code{z}, \code{q}, and \code{v}. On line 6, we start the linking structure. In order to create a link, the compiler must determine which variable will be the initial or \textbf{root} node for the link. The compiler will interpret line 6 to mean that the variable \code{x} is being linked to one or more variables, and to initialize \code{x} as the root of the link tree. Lines 7 through 11 introduce the nested structures of \textbf{link blocks}. As we traverse deeper into the nesting of the link block, we can create a \textbf{link tree} that shows how the linking structure is suppose to work. The \textbf{link tree} for this code is shown in Figure 1.

\begin{wrapfigure}{r}{0.5\textwidth}
\centering
\begin{tikzpicture}[->,>=stealth',shorten >=1pt,auto,node distance=2.5cm]
\node[state] (x)                    {$\code{x}$};
\node[state] (y) [below right of=x] {$\code{y}$};
\node[state] (z) [below left of=y]  {$\code{z}$};
\node[state] (v) [below right of=y] {$\code{v}$};
\node[state] (q) [below left of=x]  {$\code{q}$};
\path[->] (x) edge node {} (y);
\path[->] (x) edge node {} (q);
\path[->] (y) edge node {} (z);
\path[->] (y) edge node {} (v);
\end{tikzpicture}
\caption{The link tree for \emph{sample.rpl}}
\end{wrapfigure}

As we traverse down the link tree, we see \code{y} links to \code{x}. Thus, we create a child node of \code{x} on the link tree and name it \code{y}. Moving deeper into the nesting, we see that there are two other links (\code{z} and \code{v}) that link to \code{y}. These two variables will become children nodes of \code{y}. Finally, we leave the inner link block and discover the link to \code{x} by \code{q}; \code{q} now becomes another child node for \code{x}.

All links are hierarchical and uni-directional in nature. For example, if the value of \code{y} from our previous example were to change, only the children of \code{y} would be affected (\code{z} and \code{v}). Siblings, parents, and ancestors of the updated node will not be notified of the update. 

Furthermore, link statements with any type of \code{StreamReader} need not initialize a root node for the link tree. Since all \code{StreamReader}s read in volatile external data, the compiler is able to infer that the variable linked to the \code{StreamReader} will be the child of the specified \code{StreamReader}.

Link statements are syntactically similar to \code{if}-statements; thus, allowing
programmers to link a specific variable to a data stream or to another variable.
Developers can also provide functions to be executed every time the left hand
side variable is updated. Links can be made between two lists of equal length
through a one-to-one mapping of indices within the lists.

\code{StreamReader} is Ripple's construct for allowing users to define data
streams, and serves to specify how data will enter the program. A
\code{StreamReader} consists of two methods: \code{OpenStream()}, which
facilitates any initial setup of the stream before it can be read (e.g.
establishing an HTTP connection), and \code{ReadStream()}, which returns the
next piece of data from available from the stream. Common \code{StreamReader}s,
such as for files, keyboard input, and XML documents, are provided in Ripple's
standard library, and others can be defined by the programmer, maximizing
flexibility.
    
Functions are defined using the keyword \code{func}. Every function can take a variable number of arguments, and uses the keyword \code{returns} with a data type to represent what the function returns. If there is no returns statement present, then the function returns nothing.


\end{document}
